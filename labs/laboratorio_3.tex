\documentclass[a4paper,12pt,final]{article}
\usepackage[scaled=0.9]{luximono}
\usepackage[spanish]{babel}
\usepackage[utf8]{inputenc}
\usepackage[T1]{fontenc}
\usepackage{booktabs}
\usepackage{enumitem}
\usepackage{epstopdf}
\usepackage{floatrow}
\usepackage{geometry}
\usepackage{graphicx}
\usepackage{hyperref}
\usepackage{listings}
\usepackage{multicol}
\usepackage{tabularx}
\usepackage{textcomp}
\usepackage{amsmath}
\usepackage{amssymb}
\usepackage{amstext}
\usepackage{caption}
\usepackage{charter}
\usepackage{wrapfig}
\usepackage{amsbsy}
\usepackage{amsthm}
\usepackage{lipsum}
\usepackage{minted}
\usepackage{natbib}
\usepackage{array}
\usepackage{color}
\usepackage{esint}
\usepackage{float}

% Hyperref setup
\hypersetup{
  pdftitle={Procesamiento de datos digitales. Laboratorio 3},
  pdfauthor={Martín Josemaría Vuelta Rojas},
  pdfpagelayout=OneColumn,
  pdfnewwindow=true,
  pdfdisplaydoctitle=true,
  pdfstartview=XYZ,
  plainpages=false,
  unicode=true,
  bookmarksnumbered=true,
  bookmarksopen=true,
  bookmarksopenlevel=3,
  breaklinks=true,
  colorlinks=true,
  linkcolor=blue,
  pdfborder={0 0 0}
}

% Minted settings
\setminted[matlab]{
  autogobble=true,
  linenos=false,
  bgcolor=grey_lighten_4,
  fontfamily=\ttdefault,
  resetmargins=true,
  stripnl=true,
  breaklines=true
  breakautoindent=true,
  breaksymbolleft=\tiny\ensuremath{\hookrightarrow},
  breaksymbolright=\tiny\ensuremath{\hookleftarrow},
  fontsize=\footnotesize
}

\setminted[javascript]{
  autogobble=true,
  linenos=false,
  bgcolor=grey_lighten_4,
  fontfamily=\ttdefault,
  resetmargins=true,
  stripnl=true,
  breaklines=true
  breakautoindent=true,
  breaksymbolleft=\tiny\ensuremath{\hookrightarrow},
  breaksymbolright=\tiny\ensuremath{\hookleftarrow},
  fontsize=\footnotesize
}

\setminted[text]{
  autogobble=true,
  linenos=false,
  bgcolor=grey_lighten_4,
  fontfamily=\ttdefault,
  resetmargins=true,
  stripnl=true,
  breaklines=true
  breakautoindent=true,
  breaksymbolleft=\tiny\ensuremath{\hookrightarrow},
  breaksymbolright=\tiny\ensuremath{\hookleftarrow},
  fontsize=\footnotesize
}

\geometry{
  a4paper,
  total={210mm,297mm},
  left=20mm,
  right=20mm,
  top=20mm,
  bottom=20mm,
}

\floatsetup[listing]{
  capposition=top,
  style=ruled,
}

\captionsetup[listing]{
  labelfont=bf,
  justification=centering
}

\floatsetup[figure]{
  capposition=top,
  style=ruled,
}

\floatsetup[wrapfigure]{
  capposition=top,
  style=plain,
}

\captionsetup[figure]{
  labelfont=bf,
  justification=centering
}

%% LaTeX commands.
\makeatletter
%% -----------------------------------------------------------------------------
\definecolor{grey_lighten_4}{rgb}{0.9804, 0.9804, 0.9804}
%% Caption name for minted environments
% \SetupFloatingEnvironment{listing}{name=Script}
% \SetupFloatingEnvironment{listing}{listname=Lista de scripts}
\renewcommand{\listingscaption}{Script}
\renewcommand{\listoflistingscaption}{Lista de scripts}
%% Redefinition of \maketitle command
\def\@maketitle{%
  \newpage%
  \null%
  \vskip 0em%
  \begin{flushleft}%
      \let \footnote \thanks%
      {\LARGE \@title \par}%
  \end{flushleft}%
  \begin{flushright}%
      \vskip 1em%
      {\@author \par}%
  \end{flushright}%
  \noindent\rule{1\columnwidth}{1pt}%
  \par%
}

\makeatother

%% -----------------------------------------------------------------------------
\begin{document}
  \title{\textit{\Large Laboratorio Nº3}\linebreak{}\linebreak{}\textbf{\Huge Convolución y respuesta en el tiempo}}
  \author{\emph{Martín Josemaría Vuelta Rojas}}
  \maketitle

  \subsection*{Problema 1}
    \noindent Sean
    $$x\left(n\right) = \delta\left(n\right) + 2\delta\left(n-1\right) - \delta\left(n-3\right)$$
    \noindent y
    $$h\left(n\right) = 2\delta\left(n+1\right) + 2\delta\left(n-1\right).$$
    Calcule y haga la gráfica (usar el comando \texttt{stem}) de cada una de las siguientes
    convoluciones:

    \begin{enumerate}[label=\alph*)]
      \item $y_1\left(n\right) = x\left(n\right)*h\left(n\right)$
      \item $y_2\left(n\right) = x\left(n+2\right)*h\left(n\right)$
      \item $y_3\left(n\right) = x\left(n\right)*h\left(n+2\right)$
    \end{enumerate}

    \subsubsection*{Solución}
      \begin{listing}[H]
        \caption{Función impulso unitario}
        \label{script01A}
        \inputminted{matlab}{./laboratorio_3/impulso.m}
      \end{listing}

      \begin{listing}[H]
        \caption{Función $x\left(n\right)$}
        \label{script01B}
        \inputminted{matlab}{./laboratorio_3/p1_X.m}
      \end{listing}

      \begin{listing}[H]
        \caption{Función $h\left(n\right)$}
        \label{script01C}
        \inputminted{matlab}{./laboratorio_3/p1_H.m}
      \end{listing}

      \begin{listing}[H]
        \caption{Convoluciones $x\left(n\right)*h\left(n\right)$, $x\left(n+2\right)*h\left(n\right)$ y $x\left(n\right)*h\left(n+2\right)$ en \textsc{Matlab} }
        \label{script01D}
        \inputminted{matlab}{./laboratorio_3/problema01.m}
      \end{listing}

      \begin{listing}[H]
        \caption{Resultados de ejecutar el \emph{script \ref{script01D}}}
        \label{script01E}
        \begin{minted}{matlab}
          >> problema01
          x(n)*h(n)  :  0  0  0  0  0  0  0  0  0  2  4  2  2  0 -2  0  0  0  0  0  0
          x(n+2)*h(n):  0  0  0  0  0  0  0  2  4  2  2  0 -2  0  0  0  0  0  0  0  0
          x(n)*h(n+2):  0  0  0  0  0  0  0  2  4  2  2  0 -2  0  0  0  0  0  0  0  0
        \end{minted}
      \end{listing}

      \begin{figure}[H]
        \caption{Gráfico de la función impulso unitario del \emph{script \ref{script01A}}.}
        \label{script01Afigure}
        \includegraphics[width=0.90\textwidth]{./laboratorio_3/problema01_impulso.png}
      \end{figure}

      \begin{figure}[H]
        \caption{Gráfico de la función $x\left(n\right)$ del \emph{script \ref{script01B}}.}
        \label{script01Bfigure}
        \includegraphics[width=0.90\textwidth]{./laboratorio_3/problema01_x.png}
      \end{figure}

      \begin{figure}[H]
        \caption{Gráfico de la función $h\left(n\right)$ del \emph{script \ref{script01C}}.}
        \label{script01Cfigure}
        \includegraphics[width=0.90\textwidth]{./laboratorio_3/problema01_h.png}
      \end{figure}

      \begin{figure}[H]
        \caption{Gráfico de la función $y_1\left(n\right)=x\left(n\right)*h\left(n\right)$ calculada \emph{script \ref{script01D}}.}
        \label{script01Dfigure}
        \includegraphics[width=0.90\textwidth]{./laboratorio_3/problema01_y1.png}
      \end{figure}

      \begin{figure}[H]
        \caption{Gráfico de la función $y_2\left(n\right)=x\left(n+2\right)*h\left(n\right)$ calculada \emph{script \ref{script01D}}.}
        \label{script01Efigure}
        \includegraphics[width=0.90\textwidth]{./laboratorio_3/problema01_y2.png}
      \end{figure}

      \begin{figure}[H]
        \caption{Gráfico de la función $y_3\left(n\right)=x\left(n\right)*h\left(n+2\right)$ calculada \emph{script \ref{script01D}}.}
        \label{script01Ffigure}
        \includegraphics[width=0.90\textwidth]{./laboratorio_3/problema01_y3.png}
      \end{figure}

      \noindent De forma analítica, obtenemos las convoluciones solicitadas empleando la definición:
      $$y\left(m\right) = x\left(n\right)*h\left(n\right) = \sum_n x\left(n\right)h\left(m-n\right)$$
      \noindent Así obtenemos

      \begin{equation*}
        \begin{split}
          y_1\left(m\right) & = x\left(n\right)*h\left(n\right) \\
                            & = \sum_n \left[\delta\left(n\right) + 2\delta\left(n-1\right) - \delta\left(n-3\right)\right]
                                       \left[2\delta\left(m-n+1\right) + 2\delta\left(m-n-1\right)\right] \\
                            & = 2\delta\left(m+1\right) + 4\delta\left(m\right) + 2\delta\left(m-1\right) + 2\delta\left(m-2\right) - 2\delta\left(m-4\right) \\
        \end{split}
      \end{equation*}

      \begin{equation*}
        \begin{split}
          y_2\left(m\right) & = x\left(n+2\right)*h\left(n\right) \\
                            & = \sum_n \left[\delta\left(n+2\right) + 2\delta\left(n+1\right) - \delta\left(n-1\right)\right]
                                       \left[2\delta\left(m-n+1\right) + 2\delta\left(m-n-1\right)\right] \\
                            & = 2\delta\left(m+3\right) + 4\delta\left(m+2\right) + 2\delta\left(m+1\right) + 2\delta\left(m\right) - 2\delta\left(m-2\right) \\
        \end{split}
      \end{equation*}

      \begin{equation*}
        \begin{split}
          y_3\left(n\right) & = x\left(n\right)*h\left(n+2\right) \\
                            & = \sum_n \left[\delta\left(n\right) + 2\delta\left(n-1\right) - \delta\left(n-3\right)\right]
                                       \left[2\delta\left(m-n+3\right) + 2\delta\left(m-n+1\right)\right] \\
                            & = 2\delta\left(m+3\right) + 4\delta\left(m+2\right) + 2\delta\left(m+1\right) + 2\delta\left(m\right) - 2\delta\left(m-2\right) \\
        \end{split}
      \end{equation*}

  \newpage
  \subsection*{Problema 2}
    \noindent Considere un sistema LIT cuya respuesta a la señal $x_1\left(t\right)$ es $y_1\left(t\right)$

    \begin{figure}[H]
      \begin{center}
        \includegraphics[width=0.45\textwidth]{./laboratorio_3/problema02_X1.png}
        \includegraphics[width=0.45\textwidth]{./laboratorio_3/problema02_Y1.png}
      \end{center}
    \end{figure}

    \noindent Hallar las respuestas del sistema anterior a las siguientes excitaciones:

    \begin{figure}[H]
      \begin{center}
        \includegraphics[width=0.45\textwidth]{./laboratorio_3/problema02_X2.png}
        \includegraphics[width=0.45\textwidth]{./laboratorio_3/problema02_X3.png}
      \end{center}
    \end{figure}

    \subsubsection*{Solución}
      \noindent Las funciones $x_2\left(t\right)$ y $x_3\left(t\right)$ se pueden representar en función de $x_1\left(t\right)$ como
      $$x_2\left(t\right) = x_1\left(t\right) + x_1\left(t - 1\right)$$
      \noindent y
      $$x_3\left(t\right) = x_1\left(t\right) - x_1\left(t - 2\right).$$

      De forma que si consideramos la función de transferencia del sistema como $h\left(t\right)$ y
      $$y_1\left(t\right) = x_1\left(t\right) * h\left(t\right),$$
      \noindent entonces
      $$y_2\left(t\right) = x_1\left(t\right) * h\left(t\right) + x_1\left(t - 1\right) * h\left(t\right) = y_1\left(t\right) + y_1\left(t - 1\right)$$
      \noindent y
      $$y_3\left(t\right) = x_1\left(t\right) * h\left(t\right) - x_1\left(t - 2\right) * h\left(t\right) = y_1\left(t\right) - y_1\left(t - 2\right).$$
      \noindent Con estas observaciones escribimos las soluciones a este problema en \textsc{MatLab}

      \begin{listing}[H]
        \caption{Función $x_1\left(t\right)$}
        \label{script02A}
        \inputminted{matlab}{./laboratorio_3/p2_X1.m}
      \end{listing}

      \begin{listing}[H]
        \caption{Función $y_1\left(t\right)$}
        \label{script02B}
        \inputminted{matlab}{./laboratorio_3/p2_Y1.m}
      \end{listing}

      \begin{listing}[H]
        \caption{Función $x_2\left(t\right)$}
        \label{script02C}
        \inputminted{matlab}{./laboratorio_3/p2_X2.m}
      \end{listing}

      \begin{listing}[H]
        \caption{Función $y_2\left(t\right)$}
        \label{script02D}
        \inputminted{matlab}{./laboratorio_3/p2_Y2.m}
      \end{listing}

      \begin{listing}[H]
        \caption{Función $x_3\left(t\right)$}
        \label{script02E}
        \inputminted{matlab}{./laboratorio_3/p2_X3.m}
      \end{listing}

      \begin{listing}[H]
        \caption{Función $y_3\left(t\right)$}
        \label{script02F}
        \inputminted{matlab}{./laboratorio_3/p2_Y3.m}
      \end{listing}

      \begin{figure}[H]
          \caption{Función $x_2\left(t\right)$ y respuesta $y_2\left(t\right)$ empleando los \emph{scripts \ref{script02C}} y \emph{\ref{script02D}}.}
          \label{script02Afigure}
          \begin{center}
              \includegraphics[width=0.45\textwidth]{./laboratorio_3/problema02_X2.png}
              \includegraphics[width=0.45\textwidth]{./laboratorio_3/problema02_Y2.png}
          \end{center}
      \end{figure}

      \begin{figure}[H]
          \caption{Función $x_3\left(t\right)$ y respuesta $y_3\left(t\right)$ empleando los \emph{scripts \ref{script02E}} y \emph{\ref{script02F}}.}
          \label{script02Bfigure}
          \begin{center}
              \includegraphics[width=0.45\textwidth]{./laboratorio_3/problema02_X3.png}
              \includegraphics[width=0.45\textwidth]{./laboratorio_3/problema02_Y3.png}
          \end{center}
      \end{figure}
      \vspace{\fill}

  \newpage
  \subsection*{Problema 3}
    \noindent Calcular la convolución entre los siguientes pares de señales:
    \begin{enumerate}[label=\alph*)]
      \item $x\left(n\right) = \left(\frac{1}{2}\right)^n u\left(n-4\right)$ y $h\left(n\right) = 4^n u\left(2-n\right)$
      \item $x\left(n\right) = u\left(-n\right) - u\left(-n-2\right)$ y $h\left(n\right) = u\left(n-1\right) - u\left(n-4\right)$
      \item $x\left(n\right) = u\left(n\right)$ y $h\left(n\right) = \left(\frac{1}{2}\right)^{-n} u\left(-n\right)$
      \item $x\left(t\right) = \exp\left(-at\right) u(t)$ y $h(t) = exp(-at) u(t)$
    \end{enumerate}

    \noindent Donde $u\left(n\right)$ es la función escalón unitario.

    \subsubsection*{Solución}
      \begin{enumerate}[label=\alph*)]
        \item $x\left(n\right) = \left(\frac{1}{2}\right)^n u\left(n-4\right)$ y $h\left(n\right) = 4^n u\left(2-n\right)$

          \begin{equation*}
            \begin{split}
              y\left(m\right) & = x\left(n\right) * h\left(n\right) \\
                              & = \sum_{n=-\infty}^{\infty} x\left(n\right)h\left(m-n\right) \\
                              & = \sum_{n=-\infty}^{\infty} \left(\frac{1}{2}\right)^n u\left(n-4\right)4^{m-n} u\left(2-m+n\right) \\
                              & = \sum_{n=-\infty}^{\infty} 2^{2m-3n} u\left(n-4\right) u\left(n+2-m\right) \\
            \end{split}
          \end{equation*}

          \noindent Cuando $m<6$:

          \begin{equation*}
            \begin{split}
              y\left(m\right) & = \sum_{n=-\infty}^{\infty} 2^{2m-3n} u\left(n-4\right) u\left(n+2-m\right) \\
                              & = \sum_{n=-\infty}^{\infty} 2^{2m-3n} u\left(n-4\right) \\
                              & = \sum_{n=4}^{\infty} 2^{2m-3n} \\
                              & = \sum_{n=0}^{\infty} 2^{2m-3\left(n+4\right)} \\
                              & = \sum_{n=0}^{\infty} 2^{2m-3n-12} \\
                              & = 2^{2m-12} \sum_{n=0}^{\infty} \left(2^{-3}\right)^{n} \\
                              & = 2^{2m-12} \sum_{n=0}^{\infty} \left(\frac{1}{8}\right)^{n} \\
                              & = 2^{2m-12} \left(\frac{1}{1-\frac{1}{8}}\right) \\
                              & = \frac{2^{2m-9}}{7}
            \end{split}
          \end{equation*}
          \vfill
          \newpage

          \noindent Cuando $m \geq 6$:

          \begin{equation*}
            \begin{split}
              y\left(m\right) & = \sum_{n=-\infty}^{\infty} 2^{2m-3n} u\left(n-4\right) u\left(n+2-m\right) \\
                              & = \sum_{n=-\infty}^{\infty} 2^{2m-3n} u\left(n+2-m\right) \\
                              & = \sum_{n=m-2}^{\infty} 2^{2m-3n} \\
                              & = \sum_{n=0}^{\infty} 2^{2m-3\left(n+m-2\right)} \\
                              & = \sum_{n=0}^{\infty} 2^{6-m-3n} \\
                              & = 2^{6-m} \sum_{n=0}^{\infty} \left(2^{-3}\right)^{n} \\
                              & = 2^{6-m} \sum_{n=0}^{\infty} \left(\frac{1}{8}\right)^{n} \\
                              & = 2^{6-m} \left(\frac{1}{1-\frac{1}{8}}\right) \\
                              & = \frac{2^{9-m}}{7}
            \end{split}
          \end{equation*}

          \noindent Finalmente

          \begin{equation*}
            y\left(m\right) = \left\{
              \begin{array}{lcl}
                \displaystyle{\frac{2^{2m-9}}{7}} &;& n < 6 \\ [1em]
                \displaystyle{\frac{2^{9-m}}{7}}  &;& n \geq 6
              \end{array}
            \right.
          \end{equation*}

        \item $x\left(n\right) = u\left(-n\right) - u\left(-n-2\right)$ y $h\left(n\right) = u\left(n-1\right) - u\left(n-4\right)$
          \begin{equation*}
            \begin{split}
              y\left(m\right) & = x\left(n\right) * h\left(n\right) \\
                              & = \sum_{n=-\infty}^{\infty} x\left(n\right)h\left(m-n\right) \\
                              & = \sum_{n=-\infty}^{\infty} \left[u\left(-n\right) - u\left(-n-2\right)\right]
                                                            \left[u\left(m-n-1\right) - u\left(m-n-4\right)\right] \\
                              & = \sum_{n=-\infty}^{\infty} u\left(-n\right)u\left(m-n-1\right) - \sum_{n=-\infty}^{\infty} u\left(-n-2\right)u\left(m-n-1\right) - \\
                              & \phantom{=}\ \,\sum_{n=-\infty}^{\infty} u\left(-n\right)u\left(m-n-4\right) + \sum_{n=-\infty}^{\infty} u\left(-n-2\right)u\left(m-n-4\right)
            \end{split}
          \end{equation*}
          \vfill
          \newpage

          \noindent Cuando $m<0$:

          \begin{equation*}
            \begin{split}
              y\left(m\right) & = \sum_{n=-\infty}^{m-1} 1 -
                                  \sum_{n=-\infty}^{m-1} 1 -
                                  \sum_{n=-\infty}^{m-4} 1 +
                                  \sum_{n=-\infty}^{m-4} 1 \\
                              & = 0
            \end{split}
          \end{equation*}

          \noindent Cuando $m=0$:

          \begin{equation*}
            \begin{split}
              y\left(m\right) & = \sum_{n=-\infty}^{-1} 1 -
                                  \sum_{n=-\infty}^{-2} 1 -
                                  \sum_{n=-\infty}^{-3} 1 +
                                  \sum_{n=-\infty}^{-3} 1 \\
                              & = 1
            \end{split}
          \end{equation*}

          \noindent Cuando $0<m \wedge m<3$:

          \begin{equation*}
            \begin{split}
              y\left(m\right) & = \sum_{n=-\infty}^{0} 1 -
                                  \sum_{n=-\infty}^{-2} 1 -
                                  \sum_{n=-\infty}^{-2} 1 +
                                  \sum_{n=-\infty}^{-2} 1 \\
                              & = 2
            \end{split}
          \end{equation*}

          \noindent Cuando $m=3$:

          \begin{equation*}
            \begin{split}
              y\left(m\right) & = \sum_{n=-\infty}^{0} 1 -
                                  \sum_{n=-\infty}^{-2} 1 -
                                  \sum_{n=-\infty}^{-1} 1 +
                                  \sum_{n=-\infty}^{-2} 1 \\
                              & = 1
            \end{split}
          \end{equation*}

          \noindent Cuando $m>3$:

          \begin{equation*}
            \begin{split}
              y\left(m\right) & = \sum_{n=-\infty}^{0} 1 -
                                  \sum_{n=-\infty}^{-1} 1 -
                                  \sum_{n=-\infty}^{0} 1 +
                                  \sum_{n=-\infty}^{-1} 1 \\
                              & = 0
            \end{split}
          \end{equation*}

          \noindent Finalmente

          \begin{equation*}
            y\left(m\right) = \left\{
              \begin{array}{lcl}
                0 &;& m < 0  \vee  3 < m \\ [1em]
                1 &;& m = 0  \vee  m = 3 \\ [1em]
                2 &;& 0 < m \wedge m < 3
              \end{array}
            \right.
          \end{equation*}
          \vfill
          \newpage

        \item $x\left(n\right) = u\left(n\right)$ y $h\left(n\right) = \left(\frac{1}{2}\right)^{-n} u\left(-n\right)$

          \begin{equation*}
            \begin{split}
              y\left(m\right) & = x\left(n\right) * h\left(n\right) \\
                              & = \sum_{n=-\infty}^{\infty} x\left(n\right)h\left(m-n\right) \\
                              & = \sum_{n=-\infty}^{\infty} u\left(n\right) \left(\frac{1}{2}\right)^{n-m} u\left(n-m\right) \\
                              & = \sum_{n=-\infty}^{\infty} \left(\frac{1}{2}\right)^{n-m} u\left(n\right) u\left(n-m\right)
            \end{split}
          \end{equation*}

          \noindent Cuando $m < 0$:

          \begin{equation*}
            \begin{split}
              y\left(m\right) & = \sum_{n=-\infty}^{\infty} \left(\frac{1}{2}\right)^{n-m} u\left(n\right) u\left(n-m\right) \\
                              & = \sum_{n=0}^{\infty} \left(\frac{1}{2}\right)^{n-m} \\
                              & = 2^{m} \sum_{n=0}^{\infty} \left(\frac{1}{2}\right)^{n} \\
                              & = 2^{1+m}
            \end{split}
          \end{equation*}

          \noindent Cuando $0 \leq m$:

          \begin{equation*}
            \begin{split}
              y\left(m\right) & = \sum_{n=-\infty}^{\infty} \left(\frac{1}{2}\right)^{n-m} u\left(n\right) u\left(n-m\right) \\
                              & = \sum_{n=0}^{\infty} \left(\frac{1}{2}\right)^{n-m} \\
                              & = 2^{m} \sum_{n=0}^{\infty} \left(\frac{1}{2}\right)^{n} \\
                              & = 2^{1+m}
            \end{split}
          \end{equation*}

        \item $x\left(t\right) = \exp\left(-at\right) u(t)$ y $h(t) = \exp\left(-at\right) u(t)$

          \begin{equation*}
            \begin{split}
              y\left(\tau\right) & = x\left(t\right) * h\left(t\right) \\
                                 & = \int_{-\infty}^{\infty} x\left(t\right)h\left(\tau-t\right)dt \\
                                 & = \int_{-\infty}^{\infty} \exp\left(-at\right) u(t) \exp\left(-a\tau + at\right) u(t-\tau) dt\\
                                 & = \int_{0}^{\tau} \exp\left(-a\tau\right) dt \\
                                 & = \exp\left(-a\tau\right) \tau
            \end{split}
          \end{equation*}

      \end{enumerate}

  \newpage
  \subsection*{Problema 4}
    \noindent Para el diagrama de bloques mostrado
    \begin{figure}[H]
      \begin{center}
        \includegraphics[width=0.7\textwidth]{./laboratorio_3/problema04_diagram.png}
      \end{center}
    \end{figure}

    \noindent Donde

    $$h_1\left(n\right) = \beta\delta\left(n-1\right)$$

    \noindent y

    $$h_2\left(n\right) = \exp\left(\alpha\right)\delta\left(n\right)$$

    \begin{enumerate}[label=\alph*)]
      \item Escribir la ecuación en diferencias que relaciona la entrada con la salida
      \item Hallar $\alpha$ y $\beta$, de tal forma que la salida sea el promedio entre la entrada en el instante $n$ y
      la entrada en el instante $n-1$.
    \end{enumerate}

    \subsubsection*{Solución}
      \begin{enumerate}[label=\alph*)]
        \item La ecuación en diferencias se obtiene de calcular
          $$ y\left(n\right) = \left[\left(x + x*h_1\right)*h_2\right]\left(n\right) $$

          \noindent Resolviendo $\left[x*h_1\right]\left(n\right)$, tenemos

          \begin{equation*}
            \begin{split}
              y_1\left(n\right) & = x(m)*h_1(m) \\
                                & = \sum_{m=-\infty}^{\infty} x\left(m\right)h_1\left(n-m\right) \\
                                & = \sum_{m=-\infty}^{\infty} x\left(m\right)\beta\delta\left(n-m-1\right) \\
                                & = \beta x\left(n-1\right)
            \end{split}
          \end{equation*}

          \noindent Haciendo $y_2\left(n\right)=y_1\left(n\right) + x\left(n\right)$, calculamos $\left[y_2*h_2\right]\left(n\right)$ como

          \begin{equation*}
            \begin{split}
              y\left(n\right) & = y_2(m)*h_2(m) \\
                              & = \sum_{m=-\infty}^{\infty} y_2(m)h_2\left(n-m\right) \\
                              & = \sum_{m=-\infty}^{\infty} \left[y_1\left(m\right) + x\left(m\right)\right]\exp\left(\alpha\right)\delta\left(n-m\right) \\
                              & = \sum_{m=-\infty}^{\infty} \left[\beta x\left(m-1\right) + x\left(m\right)\right]\exp\left(\alpha\right)\delta\left(n-m\right) \\
                              & = \left[\beta x\left(n-1\right) + x\left(n\right)\right]\exp\left(\alpha\right)
            \end{split}
          \end{equation*}

          \noindent De modo que la ecuación en diferencias del sistema queda expresada como

          $$y\left(n\right) = \left[\beta x\left(n-1\right) + x\left(n\right)\right]\exp\left(\alpha\right)$$

        \item Para que la señal de salida sea igual al promedio de los valores de la señal de entrada en el instante $n$
          y en el instante $n-1$ se debe resolver el sistema

          \begin{equation}
            \left\{
              \begin{array}{ccc}
                \exp\left(\alpha\right)      &=&  \displaystyle{\frac{1}{2}} \\[1em]
                \exp\left(\alpha\right)\beta &=&  \displaystyle{\frac{1}{2}}
              \end{array}
            \right.
          \end{equation}

          \noindent De donde resulta

          $$ \alpha = \ln\left(\frac{1}{2}\right) \wedge \beta = 1$$
      \end{enumerate}

  \newpage
  \subsection*{Problema 5}
    \noindent Dada la siguiente ecuación en diferencias
    $$y\left(n\right) = -a y\left(n-1\right) + b x\left(n\right) + c x\left(n-1\right),$$
    realizar una representación en diagrama de bloques.

    \subsubsection*{Solución}

  \newpage
  \subsection*{Problema 6}
    \noindent Realizar en \textsc{Matlab} la convolución del siguiente par de
    señales:

    \begin{enumerate}
        \item $x\left(n\right) = \left(-1\right)^n \left(u\left(n\right) - u\left(-n-8\right)\right)$
        \item $h\left(n\right) = u\left(n\right) - u\left(n-8\right)$
    \end{enumerate}

    Graficar la señal resultante, $y\left(n\right) = x\left(n\right) * h\left(n\right)$.
    Usar el comando \texttt{stem}.

    \subsubsection*{Solución}

  \newpage
  \subsection*{Problema 7}
    \noindent Considere un sistema lineal e invariante en el tiempo, causal,
    cuya entrada $x\left(n\right)$ y salida $y\left(n\right)$ estén
    relacionadas por la ecuación de diferencias:
    $$y\left(n\right) = 0.25 y\left(n-1\right) + x\left(n\right)$$
    Determine $y\left(n\right)$ si $x\left(n\right) = \delta\left(n-1\right)$.
    Grafique en \textsc{Matlab} la salida $y\left(n\right)$, use el comando \texttt{stem}.

    \subsubsection*{Solución}

\end{document}
